\documentclass[12pt, twocolumn]{journal}
\begin{document}
\title{Machine Learning approaches for Smart Grid : A review}
\author{Aviral Mangal, Faculty of Engineering, Dayalbagh Educational Institute, Agra, India}
\begin{abstract}
 The next generation power grid, or the improved version of conventional grid, as it may be called, smart grid has become a major research interest due to depleting non renewable energy resources, the alarming increase in pollution level, and the problems that people face with the conventional grids. It is because of the many advantages that smart grid hold over conventional grids, that it is rapidly taking place of the latter. The impetus behind the success of smart grid would lie in the approaches taken to make the grid smart. Up till now, a lot of work has been done using data analytics, machine learning, fuzzy logic, soft computing, and quantum computing. In this paper, we survey the literature from 2010- 2016 on different machine learning procedures that have been implemented inn smart grid. We also explore the role of quantum mechanics in improving the performance of conventional grids. Finally, the current challenges in machine learning approaches for smart grid and the future development methods with focus on cognitive computing are proposed.
\end{abstract}
\begin{keywords}
smart grids(SGs), machine learning, quantum mechanics, cognitive computing.
\end{keywords}
\section{INTRODUCTION}
\PARstart{W}{ith}
\end{document}