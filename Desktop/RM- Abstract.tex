\documentclass[12pt, a4paper]{report}
\begin{document}
\begin{abstract}
Smart grids have changed the way conventional grids used to work, i.e., it has modernized the power generation, transmission and its distribution to the end users. It has dealt efficiently with the 
problems that once existed with the centralized grids by decentralizing the power generation and ensured two way flow of electricity and information between the power generation plant and the appliance, and all points that exist in between them. Many concepts of the smart grid like distributed generation, Vehicle to Grid(V2G), Grid  to Vehicle(G2V), Machine to Machine (M2M) communication, Automatic Metering Infrastructure(AMI) leading to smart grids, dynamic pricing, load forecasting, and demand management, have significantly effected the relation between power utilities and end consumers. 

In this report, the previous literature surveys and reviews on smart grids are analyzed. Also, a study is conducted on the previous use of machine learning techniques on smart grid and a general comparison with the other ways of grid modeling. Also, focus would be there on how cognitive computing and quantum computing techniques could improve the performance of smart grid. In the end, future prospects of the next generation smart grid are discussed and the challenges that could be faced in this transition are discussed.  
\end{abstract}
\end{document}